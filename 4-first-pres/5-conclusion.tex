\section{Additional Results}

\begin{frame}
	The paper proves a few other results, which we cover without proof.
		\begin{theorem}[Yan, Lih, Kuo, Chang (1996)]
			Suppose $\sigma : d_1,...,d_p$ is an integral sequence of $p \geq 2$ terms. Suppose $\sigma$ has $p_+$ positive terms, $p_0$ zero terms and $p_-$ negative terms. Let $\delta = 1$ if $p_+p_- > 0$ and $\delta = 0$ otherwise. Then, Then $\sigma$ is the signed degree sequence of a signed tree if and only if (T1) to (T4) hold.
			\begin{itemize}
				\item[(T1)] $\sum_{i = 1}^{p} d_i \equiv 2p - 2 \pmod 4$
				\item[(T2)] $\sum_{i = 1}^{p} \lvert d_i \rvert \le 2p - 2 - 2p_0$
				\item[(T3)] $\sum_{i = 1}^{p} \lvert d_i \rvert + 2\sum_{d_i < 0} \lvert d_i \rvert \leq 2p - 2 - 4\delta + 4p_-$
				\item[(T4)] $\sum_{i = 1}^{p} \lvert d_i \rvert + 2\sum_{d_i > 0} \lvert d_i \rvert \leq 2p - 2 - 4\delta + 4p_+$
			\end{itemize}
		\end{theorem}
	\begin{itemize}
		\item The proof in this case follows by some induction and a lot of elbow grease.
		\item This gives us a characterization of the degree sequences of signed trees.
	\end{itemize}
\end{frame}

\begin{frame}
	\begin{lemma}[Chartrand, Gavlas, Harary, Schultz (1994)]
		Let $G = (V,E)$ is a signed graph. If $q := \lVert G \rVert$, $q^+$ is the number of positive edges and $q^-$ is the number of negative edges, then $k := \sum_{v \in G} sdeg(v) \equiv 2\lVert G \rVert \mod 4$ and $q^+ = (2q+k)/4$ and $q^- = (2q-k)/4$.
	\end{lemma}
	\begin{theorem}[Yan, Lih, Kuo, Chang (1996)]
		A sequence $d_1, d_2, ..., d_p$ is the signed degree of a graph with loops if and only if $\sum_{i = 1}^{p} d_i \equiv 0 \mod 2$.
	\end{theorem}
	To prove this we use the lemma and the fact that $d_1,...,d_p$ has an even number of odd elements.
\end{frame}


\begin{frame}
	\begin{theorem}[Yan, Lih, Kuo, Chang (1996)]
		A sequence $d_1, d_2, ..., d_p$ is the signed degree of a graph with multiple edges if and only if $\sum_{i = 1}^{p} d_i \equiv 0 \mod 2$.
	\end{theorem}
	This follows from the previous lemma and the construction of $G = (V,E)$ as $V = \{ v_1, ..., v_p \}$ and 
		\begin{align*}
			E = 
				& \left\{ -d_3 + \frac{1}{2}\sum_{i = 1}^{p} d_i \text{ copies of }  v_1v_2\right\} \\
				& \cup \left\{ d_2 + d_3 - \frac{1}{2}\sum_{i = 1}^{p} d_i \text{ copies of } v_2v_3\right\} \\
				& \cup \left\{ d_1 + d_3 - \frac{1}{2}\sum_{i = 1}^{p} d_i \text{ copies of } v_1v_3\right\} \\ 
				& \cup \{ d_i \text{ copies of } v_3v_i : 4 \leq i \leq p \}
		\end{align*}
\end{frame}



\section{Conclusion}


\begin{frame}
	In summary, we set out to find a Havel-Hakimi equivalent for signed graphs. After this paper we have a complete characterization of:
		\begin{enumerate}
			\item Signed degree sequences of signed graphs.
			\item Signed degree sequences of signed tree graphs.
			\item Signed degree sequences of signed graphs with loops and signed graphs with multiple edges. In particular, signed multigraphs in general.
		\end{enumerate}
\end{frame}


\begin{frame}{A little treat.}
	\begin{problem}
		Given a fixed (partially ordered) Abelian group $(K, +, \geq)$ we can define a $K$-graph $G$ as a (finite) graph $(V,E)$ where each $e \in E$ is given an associated element of $K$ (call it $e_K$). Let $E(v) = \{ xy \in E : x = v \}$ be the set of edges incident to $v$. We may then define the $K$-degree as $K\text{-}\deg(v) = \sum_{e \in E(v)} e_K$. 
		
		Hence, given a sequence $d_1 \geq d_2 \geq ... \geq d_n$ in an Abelian group $K$, when can we find a $K$-graph $G$ with vertex set $\{ v_1, ..., v_n \}$ such that $K\text{-}\deg(v_i) = d_i$? 
	\end{problem}
\end{frame}

\begin{frame}{Thank You For Listening.}
	\centering \Huge
	\emph{Fin}
\end{frame}
